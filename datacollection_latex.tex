 \documentclass {article}
\pagenumbering {arabic}
\begin{document}
\title {CONCEPT PAPER FOR COLLECTING DATA ABOUT CHILDREN BELOW 18 YEARS WITH PHONES.}

\author  { NAHURIRA HUMPHREY     15/U/9205/PS      215016851}
\maketitle
\section {INTRODUCTION.}
This concept is aimed at collection data about children with mobile phones below the age of 18.
Data was collected from school going children and those who don’t study but own mobile phones.
Mobile phones are gadgets that were introduced in Uganda in early 1990s but have become very common in less than 30 years to the extent that most children below the age of 18 own mobile phones.
Mobile phones are used for communication and for entertainment.
\section {BACKGROUND OF THE PROBLEM.}
This data to be collected from children is to help be able to determine the number of children who are being affected by the bad effects as a result of mobile phones. Mobile phones have greatly affected our culture both adults and young people below the age of 18.Many children spend most of their time on mobile phones on social media and watching pornography. This has decreased on their time to do productive work in their communities.
\section{PROBLEM STATEMENT. }This data collection is to help identify the number of children affected as a result of bad effects of mobile phones. The data collected will enable community leaders to put sessions to sensitize parents on the bad effects of mobile phones and discourage them from buying them for their children.
\section{OBJECTIVES.}
\begin{enumerate}
\item	Increase productivity in the community.
\item	Help reduce on the bad morals learnt by children as a result of mobile phones.
\item	Children will be able to get more time for their studies than spending most of their time on mobile phones.
\item	Money used by parents to buy mobile phones for their children would be put into proper use.
\end {enumerate}
\section{METHODOLOGY. }
To know the types of mobile phones that these children can afford and what mostly they are using with their phone. Another thing is to know where these children get the money they use to buy these mobile phones. Picture of the mobile phone held by the child and the GPS Coordinates of the location of the child are also included in the data collection as a way to easily identify children below 18 years with mobile phones in the community.
\section {ANTICIPATED OUTCOMES.}
Be able to determine the number of children who own mobile phones in the community and the ways through which they were able to acquire these mobile phones.
\end{document}
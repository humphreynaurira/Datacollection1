\documentclass {article}
\pagenumbering {arabic}
\begin{document}
\title {REPORT ON USAGE OF MOBILE PHONES BY CHILDREN BELOW 18 YEARS.}

\author  { NAHURIRA HUMPHREY     15/U/9205/PS      215016851}
\maketitle

\section{Executive Summary.}
The aim of this report is to collect the data about the number of children below 18 years with mobile phones in the community. The results indicate that most children between the ages of 13-18 have mobile phones and these mobile phones were bought by their parents. They claim use these mobile phones for communicating with their parents while at school and for face book and whatsapp.
\section{Introduction.}
Mobile phones have become so common in the world due to their usage and their flexibility. They were first introduced in Uganda in early 1990 though by then they were owned by a few people and government organizations. In those days mobile phones were mostly used for just communication.
Due to the advancement in technology and increase in demand of these mobile phones and their capability and flexibility, they have become so common and to date most children below the age of 18 own a mobile phone.
\section{Methods.}
The data collection on the usage of mobile phones by children below 18 years of age was carried out by the use of questionnaire .A total of 100 questionnaire where issued out to school going children between the age of 13 and 18. It was a door to door thing and mostly children who knew how to read and write contributed towards the questionnaire.  No personal information was collected; the survey was voluntary and anonymous.  Audio messages and pictures of their phones were also required to feature in the full completed questionnaire.
\section{Results.}
Results showed that 70 of the children between the age of 13 to 18 owned mobile phones and these phones were bought by their parents to ease communication while these children are at school. The collection also showed that in the 70 most of these children were in boarding secondary schools and it was very hard to express their needs to their parents without mobile phones.
Below is a table of mobile phone usage by these teenagers.
\begin{table}[]
	\centering
	\caption{Table 1}
	\begin{tabular}{   |l | l | l|l    }
		\hline
		\bfseries { with phones} & \bfseries {usage communication in percent} & \bfseries {usage social mediain percent} \\  \hline
		Girls 74 percent & 60 & 80	\\
\hline
		Boys 66 percent& 80 &70 \\
		

\hline
	
	\end{tabular}
\end{table}



\section{Discussion.}
It can be seen from the table that more girls have phones than boys. This is because some girls have other sources of income like from their boy lovers. It can also be seen from the table that most boys use their mobile phones for communication and girls use their mobile phone mostly for social media.
\section{Conclusion.}
Parents should be strict on their children who are below 18 years of age not to hold mobile phones because this can lead to less concentration at school and hence leading to poor performance. Mobile phones can also lead to poor moral behaviors among young children because they can be able learn bad habits like immorality through watching pornography.
\section{Recommendations.}
Parents should stop their young children below the age of 18 from possessing mobile phones.
Schools should put strict measures and regulations discouraging young children from having mobile phones at school.

\end{document}
